Autoencoders are a method for performing representation learning, an unsupervised pretraining process during which a more useful representation of the
input data is automatically determined. Representation learning is important in machine learning since ``the performance of 
machine learning methods is heavily dependent on the choice of data representation (or features) in which they are applied" 
\cite{bengio2012rep}. For many supervised classification tasks, the high dimensionality of the input data means that the classifier requires an enormous number of training examples in order to generalize well and not overfit. One solution is to use unsupervised pretraining to learn a good representation for the input data and during actual training, transform the input examples into an easier form for the classifier to learn. Autoencoders are one such representation learning tool.

An autoencoder is a neural network with a single hidden layer and where the output layer and the input layer have the same size. Suppose that the input $x\in\mathbb{R}^m$ (and the output as well) and suppose that the hidden layer has $n$ nodes. Then we have a weight matrix $W\in\mathbb{R}^{m\times n}$ and bias vectors $b$ and $b^{'}$ in $\mathbb{R}^m$ and $\mathbb{R}^n$, respectively. Let $s(x) = 1/(1+e^{-x})$ be the sigmoid (logistic) transfer function. Then we have a neural network as shown in Fig.~\ref{fig:autoencoder}. When using an autoencoder to encode data, we calculate the vector $y=s(Wx + b)$; corresponding when we use an autoencoder to decode and reconstruct back the original input, we calculate $z=s(W^{T}x+b^{'})$. The weight matrix of the decoding stage is the transpose of weight matrix of the encoding stage in order to reduce the number of parameters to learn. We want to optimize $W$, $b$, and $b^{'}$ so that the reconstruction is as similar to the original input as possible with respect to some loss function. In this report, the loss function used is the least squares loss: $E(t,z)=\frac{1}{2}\norm{t-z}_2^2$, where $t$ is the original input. After an autoencoder is trained, its decoding stage is discarded and the encoding stage is used to transform the training input examples as a preprocessing step. We will refer to the trained encoding stage of the autoencoder as an ``autoencoder layer".  

For ordinary autoencoders, we usually want that $n<m$ so that the learned representation of the input exists in a lower dimensional space than the input. This is done to ensure that the autoencoder does not learn a trivial identity transformation. However, there also exists an autoencoder variant called \textit{denoising autoencoders} that use a different reconstruction criterion to learn overcomplete representations \cite{vincent2010stacked}. In other words, even if $n>m$, a denoising autoencoder can still learn a good representation of the input. This is achieved by corrupting the input image and training the autoencoder to reconstruct the original uncorrupted image. By learning how to denoise, the autoencoder is forced to understand the true structure of input data and learn a good representation of it. 

We will consider training denoising autoencoders with stochastic gradient descent (SGD). Background literature for denoising autoencoders can be found in the related work section. The algorithm description section contains more details about how SGD is implemented for autoencoders. The experiments section describes the results of training an autoencoder on a handwritten digit image dataset. Finally in future work, we discuss our plans to improve upon existing training algorithms.

